\documentclass{report}
\usepackage{polski}
\usepackage[utf8]{inputenc}
\usepackage{hyperref}
\usepackage{cite}
\usepackage{url}
%Spell check: aspell -t -l en -c ./zai.tex

\title{
	Sklepy internetowe
}
\author{
	Łukasz Wieczorek\\
	\begin{tabular}{rl}
	E-mail & \texttt{wieczorek1990@gmail.com}\\
	\end{tabular}
}

\begin{document}
\maketitle
\section{Definicja}
\emph{,,Sklep internetowy -- serwis internetowy dający możliwość zamawiania produktów przez Internet, jedna z form handlu elektronicznego.''}\cite{wiki:sklep}
Sklep internetowy powinien posiadać:
\begin{itemize}
\item katalog produktów,
\item wirtualny koszyk,
\item formularz zawierania transakcji,
\item panel administracyjny dla pracowników sklepu,
\item opcjonalnie: system raportowania.
\end{itemize}
Przeciętna ścieżka klienta w sklepie internetowym wygląda tak:
\begin{itemize}
\item odwiedza stronę główną sklepu,
\item wybiera produkty,
\item wybiera metodę płatności,
\item akceptuje zamówienie.
\end{itemize}
Największy wpływ na konfigurację sklepu internetowego mają środki pieniężne. Zakładając sklep internetowy powinniśmy odpowiedzieć sobie na pytania:
\begin{itemize}
\item Jaki mamy budżet?
\item Jaki jest profil naszej aplikacji?
\item Czy potrzebujemy własnego serwera, czy też zdamy się na zewnętrzny hosting?
\item Czy oprogramowanie sklepu będzie pisane na zamówienie, czy skorzystamy z gotowych (płatnych lub darmowych) rozwiązań?
\item Ilu użytkowników się spodziewamy?
\item Czy system będzie generował skomplikowane raporty?
\item Jaką formę marketingu wykorzystamy?
\item Jak będzie wyglądać obsługa reklamacji?
\item Jakie technologie wykorzystamy?
\item Jaką wartość dodaną posiada nasz produkt?
\end{itemize}
Niektóre z darmowych rozwiązań dla e-commerce to: Magento, osCommerce, OpenCart, SpreeCommerce, PrestaShop, Shopify, Zen Cart.
Zalety dla:
\begin{itemize}
\item klientów:
\begin{itemize}
\item oszczędność czasu
\item wygoda
\item możliwość dostępności szczegółowych informacji o produkcie
\item dostęp szerokiego asortymentu
\item dostępność oferty 24/7
\item proste wyszukiwanie
\item dostępność materiałów multimedialnych
\end{itemize}
\item firm:
\begin{itemize}
\item redukcja kosztów pośrednictwa
\item nowi klienci
\item rynek światowy
\item klient jest w ,,kontrolowanym środowisku''
\end{itemize}
\end{itemize}
Niektóre formy płatności w sklepach internetowych
\begin{itemize}
\item pobranie
\item czek
\item karta debetowa
\item karty podarunkowe
\item płatność pocztowa
\item waluty wirtualne (np. Bitcoin)
\end{itemize}

\section{Relacje w biznesie}
Ze względu na to komu sprzedajemy nasze towary nasz sklep będzie działał w formule Business to Client, bądź Business to Business.
\subsection{Business to Business}
To ogół relacji między firmą a partnerami, pośrednikami, dostawcami, dystrybutorami oraz punktami
sprzedaży i świadczenia usług wykorzystujący środki elektroniczne do zawierania transakcji. B2B są to
transakcje kupna-sprzedaży dóbr i usług w obrocie między firmami za pomocą m.in.: katalogów produktów, aliansów zakupowych, rynków barterowych, pionowych. Głównym zadaniem B2B jest integracja łańcuchów dostaw.
\subsection{Business to Client}
Polega na sprzedaży produktów przez firmy konsumentom. Obejmuje wszelkie aspekty elektronicznego biznesu na płaszczyźnie kontaktów z indywidualnymi klientami poprzez: aukcje internetowe, sklepy internetowe, sprzedaż detaliczną usług. Zadaniami B2C są zdobywanie nabywców, zapewnienie wygodnego interfejsu, stworzenie wirtualnego miejsca sprzedaży produktów firmy.
\subsection{Różnice}
\begin{itemize}
\item skala transakcji (większa na rynku przedsiębiorstw)
\item liczba nabywców (większa na rynku konsumenckim)
\item jakość informacji o produktach (większa na rynku przedsiębiorstw; logika vs emocje)
\end{itemize}
\section{Zarządzanie relacjami}
Zarządzanie relacjami na poziomie klientów i partnerów pozwala w efekcie uzyskać większe zyski, stąd ich ważność.
\subsection{Customer Relationship Management}
CRM to: \emph{,,zestaw procedur i narzędzi istotnych w zarządzaniu kontaktami z klientami.''}\cite{wiki:crm}
\subsection{Partner Relationship Management}
PRM to: \emph{,,to zarządzanie relacjami z partnerami (firmami stowarzyszonymi) poprzez zapewnienie optymalnej struktury kanału sprzedaży.''}\cite{wiki:prm}
\section{Electronic Data Interchange}
Jest to: \emph{,,transfer biznesowych informacji transakcyjnych od komputera do komputera z wykorzystaniem standardowych, zaakceptowanych formatów komunikatów.''}\cite{wiki:edi} EDI ma na celu przyspieszenie i zwiększenie dokładności danych przekazywanych pomiędzy firmami poprzez delegację zadań przekazywania informacji systemom informatycznym.
\bibliography{cite}{}
\bibliographystyle{plain}
\end{document}